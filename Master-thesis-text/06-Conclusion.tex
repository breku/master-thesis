\chapter{Wnioski}
\label{c6}

\section{Zalety programowania wizualnego}

Główną zaletą programowania wizualnego jest łatwość, z jaką przychodzi pisać aplikacje. Nie trzeba być doświadczonym programistą, aby tworzyć aplikacje za pomocą App Inventora. Jeżeli ktoś jest zainteresowany programowaniem i nie wie jak zacząć, programowanie wizualne może okazać się dobrym wyborem na start. Nauka programowania wizualnego nie jest skomplikowana, wystarczą chęci. 

Kolejną zaletą jest nieskomplikowany sposób stworzenia dobrze wyglądającego interfejsu graficznego (\english{GUI}). Nie znać języku XML aby zdefiniować ładny układ i wygląd ekranu. Wszystkie elementy są przeciągane z palety komponentów, a ich właściwości ustawiane również w prosty sposób.

Następną zaletą są możliwości jakie daje programowanie wizualne. Mimo, że wiele funkcji nie jest dostępnych to aplikacje, które można stworzyć za pomocą App Inventora mogą być bardzo rozbudowane. Komponenty oferowane przez to środowisko pokrywają zdecydowaną większość podstawowych i najbardziej używanych elementów, których się używa podczas pisania aplikacji w języku Java. Jeżeli jakiegoś komponentu brakuje, np. przycisków typu Radio, w internecie jest wiele materiałów w jaki sposób można to obejść.

Myślenie wizualne jest bardziej naturalne dla człowieka. Programowanie w App Inventorze daje programistom możliwość pisania programów poprzez manipulację elementami graficznymi. Dopiero doświadczony programista będzie potrafił sobie wyobrazić na wyższym poziomie abstrakcji kod tekstowy. Dlatego też kod aplikacji Javowej musi spełniać założenia wzorców projektowych, dzięki czemu łatwość nawigowania pomiędzy klasami i metodami jest dużo większa. W App Inventorze stworzenie zadanej funkcjonalności sprowadza się zwykle do użycia pojedynczych bloków, co daje dużą przejrzystość. Dopiero skomplikowane aplkiacje mogłyby rozrosnąć się, tak że zrozumienie ich zajełoby dużo więcej czasu niż takiej samej aplikacji napisanej w języku Java. Jednak jeżeli jest to tak skompilkowana aplikacja warto się zastanowić czy napisanie jej w App Inventorze to dobry pomysł.

App Inventor może być również dobrym narzędziem do tworzenia prototypów. Aplikacje tworzone za pomocą niego powstają bardzo szybko. Stworzony układ graficzny można w krótkim czasie zaprezentować biznesowi, który dzięki temu będzie miał szansę szybko zareagować i skorygować lub zatwierdzić dany interfejs.

\section{Wady programowania wizualnego}

Przynajmniej na razie nie ma możliwości rozszerzenia App Inventora, więc jeżeli istniałaby potrzeba stworzenia lub skorzystania z czegoś, co nie jest wbudowane bezpośrednio w oferowaną platformę, jak np. grafika 3D, to ostatecznie okaże się że projekt nie zostanie zrealizowany.

Implementacja App Inventora nie jest zoptymalizowana dla gier o wysokiej wydajności. Cały framework App Inventora zużywa o wiele więcej mocy procesora, niż programy napisane w języku Java. Przy zwykłym sortowaniu program napisany w Javie jest średnio 2 tysiące razy szybszy.

Stworzenie większego projektu i decyzja o użyciu App Inventora pociąga za sobą zaangażowanie wielu programistów. Nie mogą oni jednak pracować współbieżnie. Współdzielenie projektu odbywa się na zasadzie wyeksportowania go na komputer jako spakowane archiwum. Następnie kolejny programista może go zaimportować. Nie ma to jednak większego sensu, ponieważ kiedy dwie osoby będą pracować nad tym samym projektem, nie będzie można go na końcu scalić. Odwrotnie jest przy pisaniu aplikacji w języku natywnym. Istnieje bardzo wiele narzędzi do rozwiązania tego problemu, są to tzw. systemy zarządzania wersją kodu (\english{Source code management systems}).

\section{Inne ograniczenia App Inventora}

\begin{itemize}
\item Animacja nie jest wspierana automatycznie. Jeżeli programista chciałby stworzyć animowany GIF, posiadając kilka różnych obrazków, z których ten GIF miałby się składać, musi zrobić to manualnie. Zautomatyzowanie tego procesu byłoby bardzo pomocne.\cite{android:57}
\item App Inventor nie wspiera gestów multi-touch, czyli dotykania ekranu i wykonywanie czynności kilkoma palacmi w jednym momencie.\cite{android:57}
\item Brak wsparcia dla rysowania obrazków o standardowych kształtach. Są to między innymi prostokąty, trójkąty, koła, tekst. Programista chcąc dodać nowy element musi najpierw go stworzyć ręcznie a następnie wysłać na serwer App Inventora.\cite{android:57}
\item Niemożliwe jest tworzenie widżetów. App Inventor nie wspiera danej funkcji
\end{itemize}

