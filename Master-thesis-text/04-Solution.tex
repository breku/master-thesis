\chapter{Zastosowane podejście}
\label{c4}

W danym rozdziale zawarto opis zastosowanego podejścia, do porównania programowania wizualnego i programowania natywnego.

\section{Wstęp}
\label{c41}

Zastosowane podejście polegało na stworzeniu jak największej ilości aplikacji, wykorzystujących różne komponenty. Następnie stworzenie takich samych aplikacji w języku Java. Mając dużą liczbę aplikacji pokrywającą prawie wszystkie możliwości App Inventora, będziemy mogli odpowiedzieć na wiele pytań jego dotyczących, które zostały postawione we wstępie pracy.(\ref{c12})

\section{Dalvik Debug Monitor}

W systemie Android każda aplikacja jest uruchamiana w osobnym procesie, a każdy z procesów działa na swojej własnej wirtualnej maszynie. Każda z tych wirtualnych maszyn wystawia unikalny port, do którego może się podłączyć debbuger. Dalvik Debug Monitor (\ref{ddms}) zaraz po starcie podłącza się do Android Debug Bridge (ADB) - narzędzia, które pozwala na komunikację z podłączonym urządzeniem (\ref{adb}).

Po podłączeniu urządzenia tworzony jest serwis monitorujący pomiędzy ADB a DDMS, który powiadamia DDMS, kiedy wirtualna maszyna na urządzeniu jest uruchomiona lub zakończona. Gdy wirtualna maszyna wystartuje DDMS odbiera ID (pid) procesu uruchomionego na tej maszynie korzystając z ADB. Następnie tworzone jest połączenie do debbugera maszyny wirtualnej. Po tych operacjach DDMS jest w stanie komuniokować się z maszyną wirtualną, korzystając z dostosowanego protokołu.\cite{doc:ddms}

Poniżej widać narzędzie Dalvik Debug Monitor. Na telefonie uruchomione są dwa dodatkowe, poza systemowymi, procesy jednocześnie. Po prawej stronie widać wykres obciążenia procesora, poszczególnych procesów.

\begin{figure}[H] 
\centering\includegraphics[width=12cm]{figures/dalvik}
\caption{Przykładowy zrzut ekranu DDMS}
\end{figure}

\section{Zużycie procesora i pamięci}

Każda aplikacja powoduje zużycie procesora oraz zajmuje miejsce w pamięci. Do pomiaru tych wielkości został użyty Dalvik Debug Monitor.

Aplikację napisaną w Javie możemy konfigurować dowolnie. Między innymi możemy umożliwić aby była debugowalna, ustawiając parameter:

\begin{lstlisting}
android:debbugable="true"
\end{lstlisting}

Jest to ważne, ponieważ aplikacja (plik *.apk) wyeksportowana z App Inventora jest niemożliwa do debugowania. Powyższy parametr ma fałszywą wartość logiczną. Aby to zmienić trzeba aplikację zdekompilować, aby zobaczyć źródła aplikacji i zmienić opcję debugowania. Dekompilacja odbywa się za pomocą darmowego narzędzia apktool.

\begin{lstlisting}
apktool -d aplikacja.apk
\end{lstlisting}

Po wykonaniu powyższej komendy zostaje tworzony folder z taką samą nazwą jak nazwa aplikacji. Plik AndroidManifest.xml jest już czytelny i możemy zmienić w nim parametr odpowiadający za debugowanie. Po zmianie, aplikację trzeba skompilować ponownie. Trzeba uruchomić poniższą komendę:

\begin{lstlisting}
apktool -b aplikacja
\end{lstlisting}

Aplikacja została skompilowana ponownie do pliku *.apk. Aby zainstalować ją na urządzeniu trzeba ją jeszcze cyfrowo podpisać. Generujemy klucz dla aplikacji:

\begin{lstlisting}
keytool -genkey -v -keystore keystore -alias alias_aplikacji -keyalg RSA 
-keysize 2048 -validity 20000
\end{lstlisting}

Następnie podpisujemy aplikację:

\begin{lstlisting}
jarsigner -verbose -keystore keystore aplikacja.apk alias_aplikacji
\end{lstlisting}

Ostatecznym krokiem jest zainstalowanie aplikacji na telefonie:

\begin{lstlisting}
adb install aplikacja.apk
\end{lstlisting}

Dzięki tym wszystkim czynnościom maszyna wirtualna uruchamiająca aplikacja uruchomiona na telefonie udostępnia na port umożliwiający debugowanie. Do tego portu podłącza się Dalvik Debug Monitor, z którego możemy odczytać różne statystyki aplikacji i porównać je ze statystykami aplikacji napisanej natywnie w języku Java.
