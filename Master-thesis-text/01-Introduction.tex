\chapter{Wprowadzenie}
\label{c1}

\section{Programowanie}
\label{c11}

Definicja programowania podawana przez literaturę przedmiotu zawiera w sobie między innymi proces tworzenia aplikacji . W takim rozumieniu programowanie jest powszechnie kojarzone przez nieprofesjonalnego odbiorcę  z wielkimi ilościami kodu napisanego przez programistów. W większości przypadków tak faktycznie się dzieje. Jednak sposób pisania kodu może się różnić. Można bowiem wydzielić poziomy programowania od bardzo niskiego, wykorzystującego stosunkowo nieskomplikowane języki, do wysokiego, wymagającego wysoce specjalistycznej wiedzy i umiejętności. Przy językach niskiego poziomu, do których zalicza się np. Assembler, operowanie odbywa się na rejestrach procesora. Operacje charakteryzują się dużą szybkością, jednak napisanie bardziej skomplikowanego zadania zajęłoby ogromną ilość linii kodu oraz wymagałoby nieadekwatnego nakładu pracy. Natomiast języki wyższego poziomu wykorzystują składnię ułatwiającą zrozumienie kodu programu przez osoby, które mają z tym kodem styczność. 

Podsumowując programowanie jest procesem, który prowadzi od pierwotnego sformułowania problemu komputerowego do wykonywalnych programów. Objmuje on takie działania jak analiza, zrozumienie, ogólne rozwiązanie problemu wynikającego z algorytmu, weryfikacja wymagań algorytmu wliczając to jego poprawność i zużycie zasobów. Powszechnie określane jest jako kodowanie.\cite{android:64}\cite{android:65}

W celu jeszcze lepszego zrozumienia pisanego kodu powstała dodatkowa warstwa abstrakcji, gdzie zasadniczo nie jest wymagana od programistów ani jedna linijka kodu. Jest to programowania wizualne. Główne źródło tworzonego oprogramowania stanowią tutaj bloki graficzne i połączenia między nimi. 


\subsection{Programowanie wizualne}
\label{c111}


Programowanie wizualne jest to programowanie, które pozwala użytkownikowi tworzyć programy poprzez manipulację elementów graficznie, zatem inaczej niż w większości przypadków, gdy wykorzystywane są edytory tekstowe. Prawie wszystkie możliwe do osiągnięcia akcje mogą zostać zrealizowane tylko za pomocą myszki. 

Jednym z narzędzi, które pozwala tworzyć aplikacje wizualnie jest App Inventor. Za pomocą powyższego programu istnieje możliwość tworzenia aplikacji na system operacyjny android. Są to głównie telefony i tablety. App Inventor jest aplikacją internetową, dostępną z poziomu przeglądarki. Nie potrzebujemy dodatkowego środowiska do tworzenia programów. App Inventor jest aplikacją stworzoną przez Google, a aktualnie utrzymywaną przez uniwersytet Massachusetts Institute of Technology (MIT). Wszystkie nowe osoby, które chciałyby zacząć programować i tworzyć oprogramowanie na system operacyjny Android mogą zacząć od App Inventora. Tworzenie aplikacji jest intuicyjne dzięki graficznemu interfejsowi, który umożliwia użytkownikowi akcje typu \emph{przeciągnij i upuść} na interesujących go obiektach.\cite{wiki:appinventor} Są to proste czynności nie wymagające od użytkownika specjalistycznej wiedzy informatycznej. Nawet osoby, które nigdy nie miały do czynienia z programowaniem, nie powinny mieć większych problemów z napisaniem aplikacji. 

\subsection{Programowanie natywne}
\label{c112}

Programowanie natywne jest programowaniem na daną platformę, a więc napisane programowanie będzie na niej działać bez konieczności zainstalowania (wykorzystania, pracy) dodatkowych programów. W przypadku systemu Android jest to język Java, czyli język obiektowy wysokiego poziomu. Jest to język obiektowy wysokiego poziomu. Po napisanu programu, kod zostaje kompilowany do kodu bajtowego, którym zajmuje się maszyna wirtualna Javy (JVM). Ładuje pliki do pamięci, a następnie uruchamia zawarty w nich kod. Jednak Android nie posiada JVM. Zamiast JVM, Google wyposażył Android w maszynę Dalvik'a. Dalvik jest to maszyna wirtualna, przystosowana specjalnie do urządzeń mobilnych, gdzie szczególną uwagę należy zwrócić na małe zasoby pamięci, energii i niewielką prędkość procesorów. Kod bajtowy stworzony przez kompilator nie jest w 100\% kompatybilny z kodem bajtowym Javy. Nie można tutaj korzystać z bardziej zaawansowanych cech jakimi są Class Loadery czy Java Reflection API. \cite{gphone:dalvik}

\section{Cel i zakres pracy magisterskiej}
\label{c12}

Celem pracy magisterskiej jest porównanie tworzenia aplikacji na platformę android przy pisaniu aplikacji w języku Java oraz przy wykorzystaniu narzędzia oferowanego online - App Inventor. Praca zawiera porównanie tworzenia oprogramowania z różnych perespektyw, między innymi takich jak:
\begin{itemize}
\item Czas potrzebny na stworzenie aplikacji.
\item Możliwości jakie daje nam App Inventor, jakich rzeczy tam brakuje, a co można użyć.
\item Łatwość stworzenia aplikacji.
\item Porównanie takich samych aplikacji pod względem zużycia procesora oraz pamięci.
\item Porównanie wydajności tych samych algorytmów pod względem czasu.
\item Możliwość stworzenia bardziej zaawansowanej aplikacji korzystającej z wielu funkcji telefonu
\item Ewentualna konieczność dodatkowych narzędzi potrzebnych do tworzenia aplikacji
\end{itemize}

Dzięki takiemu porównaniu powinien wyłonić się bardziej wyrazisty obraz narzędzia, jakim jest App Inventor.  Osobom wstępnie zainteresowanym programowaniem, np.  gimnazjalistom i licealistom ułatwi decyzję odnośnie wyboru ścieżki nauki: czy warto poznać język ? App Inventor, czy też od razu (osobno!) uczyć się języka?  Java i zyskać dostęp do wszystkich funkcji Androida. Nauczanie podstaw programowania poprzez tworzenie aplikacji na system Android mogą także  rozważyć nauczyciele informatyki. 


W pracy zostały również przedstawione wady oraz zalety pisania oprogramowania przy wykorzystaniu App Inventora. Programowanie wizualne, mimo że wydaje się łatwiejsze, niesie ze sobą pewne niedogodności. Pewnych rzeczy prawdopodobnie nie da się zrealizować, a pewne są możliwe do zrealizowania w sposób o wiele prostszy. 


\subsection{Struktura pracy magisterskiej}

\begin{itemize}
\item W rozdziale \Ref{c2} przedstawiono podstawowe pojęcia zastosowane w redagowaniu pracy magisterskiej. Terminy te zostały wyjaśnione, aby bez problemu zrozumieć bardziej skomplikowane zagadnienia. Opisany jest tutaj system Android.
a także inne narzędzia, o których jest mowa w późniejszych rozdziałach. Są to między innymi SDK Tools, Jarsigner, Apktool.
Pod koniec rozdziału można znaleźć informacje o ważnych koncepcjach i pojęciach dotyczących App Inventora.
\item W rozdziale \Ref{c3} zawarto teorię dotyczącą App Inventora. Opisana jest tutaj między innymi architektura aplikacji, jego historia oraz główne komponenty, z których się składa platforma. Można również znaleźć tutaj informacje o pracy ze środowiskiem App Inventor.
\item W rozdziale \Ref{c4} pokazano zastosowane podejście do rozwiązania problemu. Opisano tutaj w jaki sposób stworzone aplikacje były testowane oraz w jaki sposób wykorzystane zostało narzędzie Dalvik Debug Monitor. Na końcu rozdziału można znaleźć informację w jaki sposób były wybierane aplikacje, które zostały stworzone.
\item W rozdziale \Ref{c5} przedstawiono wyniki uzyskane podczas pisania pracy magisterskiej. Są to stworzone aplikacje, które testują narzędzie App Inventor pod różnym kątem. Zostały tam wyzielone trzy główne sekcje: testowanie wydajności, wbudowanych elementów. Ostatnią sekcją jest próba odtworznia stworzonej poprzednio aplikacji w Javie
\item W rozdziale \Ref{c6} przedstawiono wnioski uzyskane po napisaniu pracy magisterskiej. Zestawiono  tutaj również zalety oraz wady, zarejestrowane na podstawie pisania powyższych aplikacji. Można znaleźć tutaj informacje dla kogo skierowany jest App Inventor oraz czego można od niego oczekiwać i jakie są jego ograniczenia.
\end{itemize}


















