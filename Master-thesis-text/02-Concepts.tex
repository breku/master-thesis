\chapter{Podstawowe pojęcia}
\label{c2}

W danym rozdziale zostaną zawarte podstawowe pojęcia i mechanizmy używane przez aplikację App Inventor. Ideą tutaj jest przypomnienie oraz przypliżenie ważnych terminów informatycznych.

\section{App Inventor}
\label{c21}

W grudniu 2013 roku został wydany App Inventor w wersji drugiej. Starsza wersja została nazwana jako Classic. Oba narzędzia są bardzo podobne jednak projekty stworzone w starszej wersji nie mogą zostać zaimportowane do nowszej. W danej pracy magisterskiej skupienie zostało na nowej wersji App Inventora.

\section{Główne komponenty}
\label{c22}

App Inventor celowo ułatwia programowanie poprzez wizualizację tworzonych komponentów i intuicyjny interfejs. App Inventor składa się z 3 głównych komponentów jakimi są:
\begin{itemize}
\item App Inventor Designer
\item App Inventor Blocks Editor
\item Android Device Emulator
\end{itemize}

\subsection{App Inventor Designer}
\label{c221}

Jednym z głównych widoków jakie można używać jest widok Designera. Projektowanie interfejsu użytkownika polega na przeciąganiu komponentów z dostępnej palety, wliczając w to także niewidoczne komponenty takie jak sensory. W tym widoku można również zmieniać właściwości obiektów, które stworzyliśmy. Między istnieje możliwość zmiany położenia, wielkości, układu (pionowy, poziomy) itd.

Designer jest zaprojektowany jako zwykła aplikacja internetowa. Tak więc uruchamia się go, tak jak zwykłą stronę internetową wpisując jej adres www.

\begin{figure}[th] 
\centering\includegraphics[width=10cm]{figures/designer}
\caption{App Inventor Designer}
\end{figure}

\subsection{App Inventor Blocks Editor}
\label{c222}

Drugim widokiem jest Blocks Editor. Zachowanie aplikacji zostaje tutaj zaprogramowane poprzez połączenie odpowiednich bloków. Możemy korzystać z bardziej generalnych komponentów, a także z bardziej specyficznych. Dla każdego komponentu, który został stworzony w interfejsie graficznym (Designerze) są dostępne bloki mówiące, co tak naprawdę możemy zrobić. Wygląda to tak, że przeciągamy komponenty z dostępnej palety medotą "przeciągnij i upuść", a następnie łączymy je tak jak puzzle.

Ta część aplikacji normalnie reprezentowana jest przez kod napisany przez programistę. Więc napisanie zachowania aplikacji odbywa się poprzez łączenie puzzli, bez znajomości języka Java.

\begin{figure}[th] 
\centering\includegraphics[width=10cm]{figures/editor}
\caption{App Inventor Blocks Editor}
\end{figure}

\subsection{Android Device Emulator}
\label{c223}

Android Device Emulator jest to emulator telefonu lub tabletu. Jest to wirtualna wersja smartphonu, w której znajdują się obsługa dotyku ekranu, przyciski systemowe oraz typowe funkcje.

Zmiany, które zostają wprowadzone, natychmiast reflektują na działanie aplikacji. Nie ma potrzeby jakiejkolwiek kompilacji i uruchamiania aplikacji od nowa. Jeżeli aplikacja zostanie uruchomiana, kompilacja zmienionych fragmentów oraz zainstalowanie ich na emulatorze dzieje się w czasie rzeczywistym. Jest to bardzo wygodna opcja budowania aplikacji i testowania jej. Zmiany, które zrobimy są od razu widoczne na ekranie.

\begin{figure}[th] 
\centering\includegraphics[width=10cm]{figures/emulator}
\caption{Android Device Emulator}
\end{figure}
